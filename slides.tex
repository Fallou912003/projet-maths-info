\documentclass[a4paper]{beamer}
\usepackage{amsmath, amssymb, mathtools}
\usepackage{amsthm}
\usepackage{fontspec}
\usepackage{xunicode}
\usepackage{fancyhdr}
\usepackage[french]{babel}
\usepackage{algorithm, algorithmic}
\usepackage{listings}

\usetheme{Darmstadt}
\setbeamertemplate{navigation symbols}{}
\setbeamertemplate{blocks}[]

\title[Center text]{\textbf{Compostion modulaire des polynômes à une variable}}
\subtitle{\small{Encadrant : M. Vincent NEIGER}}
\author{Serigne Fallou FALL \and Marie BONBOIRE}
\date{}


\begin{document}

\begin{frame}
    \titlepage
\end{frame}


\section{Introduction}
\begin{frame}
    \tableofcontents[currentsection]
\end{frame}

%%%%%%%%%%%%%%%%%%%%%%%%%%%%%%%%%%%%%%%%
\section{Multiplication}
\begin{frame}
    \tableofcontents[currentsection]
\end{frame}

\begin{frame}{Algorithme naïf}
    

\end{frame}


%%%%%%%%%%%%%%%%%%%%%%%%%%%%%%%%%%%%%%%%
\section{Composition modulaire des polynômes à une variable}
\begin{frame}
    \tableofcontents[currentsection]
\end{frame}

\begin{frame}[fragile]
    \begin{lstlisting}{Python}[title={Horner}]
        def horner(g, a, f) :
            res = g[g.degree()]
            for i in range(g.degree()-1, -1, -1) :
                res = (res*a)%f + g[i]
            return res%f
        \end{lstlisting}
\end{frame}

\section{Algorithme de Nusken et Ziegler}
\begin{frame}
    \tableofcontents[currentsection]
\end{frame}

\begin{frame}
    \frametitle{Algorithme de Nusken and Ziegler}
    \begin{alertblock}{Complexité Nusken}
        test
    \end{alertblock}
    \[
    \begin{pmatrix}
        g_0&g_1&...&g_{\delta -1} \\
        g_{\delta}&g_{1+\delta}&...&g_{2\delta-1} \\
        \vdots&\vdots&...&\vdots \\
        g_{(\delta-1)\delta}&g_{(\delta-1)\delta+1}&...&g_{\delta^2-1}
    
    \end{pmatrix}
    \]


\end{frame}



\end{document}
